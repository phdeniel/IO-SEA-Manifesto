\chapter{Conclusion}

The IO-SEA project had high ambitions as the proposal was written, in late 2019 and early 2020 (the deadline
to submit it to the EuroHPC~JU portal was January the 14th, 2020). As the project is now ended, the time has come
to summarize what the project actually produced.

IO-SEA has produced very concrete pieces of software. Most of them are available as open-source and can be 
downloaded at \url{https://github.com/io-sea}. This software is the result of the work from the different WP.
When looking at the back mirror, after three years, it appears clearly that the biggest strenght of the IO-SEA
has been the sinergy between the partners and the WPs. The result is an integrated software suite where each 
brick is connected with the others with well defined interface.

The main concepts supported by IO-SEA, such as the ephemeral services, the datasets the workflow manager or the
HSM features inside object stores, are key technologies for developing storage solutions fitting the requirements
of the future exascale supercomputers.

Since IO-SEA met most of its expectations, the Devil is in the details. When studying the underlying problems and
implementing the IO-SEA software, several issues were discovered. Those issues were not foreseen as the proposal
was written, and no time remained for addressing all of them. Some were partially solved, some remained untouched,
but they all were identified. Each of them is a potential track to continue developing IO software storage stack
for the exascale. 

IO-SEA has set up good basements for such a software stack. It should be leveraged to build new projects. The
related concepts should continue to involve and become more sophisticated in follow-up projects. 